\documentclass[a4paper,11pt]{article}
\usepackage[utf8]{inputenc}
\usepackage[catalan]{babel}
\usepackage{listings}
\usepackage{xcolor}
\usepackage{graphicx}
\usepackage{lscape}

\newcommand{\assignatura}{Compiladors}
\newcommand{\titol}{Hal}

\newcommand{\Pautor}{Héctor Ramón Jiménez}
\newcommand{\Sautor}{Alvaro Espuña Buxó}

\newcommand{\HRule}{\rule{\linewidth}{0.5mm}}

\newcommand{\asl}{{\tt Asl}}
\newcommand{\hal}{{\tt Hal}}

\lstdefinestyle{antlr}{
    basicstyle=\footnotesize\ttfamily,
    language=Java,
    moredelim=[s][\color{green!50!black}\ttfamily]{'}{'},% single quotes in green
    moredelim=*[s][\bf]{options}{\}},
    emph=[1]{lexer,header,parser},
    emphstyle=[1]{\color{green!70!black}\ttfamily},
    commentstyle={\color{gray}\itshape},%                  gray italics for comments
    alsoletter={:,|,;},%
    morekeywords={:,|,;},%                                 define the special characters
    keywordstyle={\color{black}\bf},%                         and format them in black
}

\lstset{
    basicstyle=\footnotesize\ttfamily,
    comment=[l]\#,
    emph=[1]{for,in,while,if,def,elif,else},
    emphstyle=[1]{\color{green!50!black}\ttfamily},
    emph=[2]{true,false},
    emphstyle=[2]{\color{blue!70!black}\ttfamily},
    commentstyle={\color{gray}\itshape},
    keywordstyle={\color{black}\bf},%                         and format them in black
}


\begin{document}

\begin{titlepage}
  \begin{center}
    \textsc{\Large \assignatura}\\[0.5cm]
    \HRule \\[0.4cm]
     { \huge \bfseries \titol \\[0.4cm] }
    \HRule \\[0.8cm]
    \begin{minipage}{0.4\textwidth}
      \begin{flushleft}
        \large \Pautor
      \end{flushleft}
    \end{minipage}
    \begin{minipage}{0.4\textwidth}
      \begin{flushright}
        \large \Sautor
      \end{flushright}
    \end{minipage}

    \vfill
    {\large \today} \\[0.3cm]
    {\large Facultat d'Informàtica de Barcelona}
  \end{center}
\end{titlepage}
\newpage\null\thispagestyle{empty}\newpage

\section{Gramàtica}

La nostra idea principal és que la sintaxi del llenguatge sigui molt neta. Per tant, el primer que vam decidir és desfer-nos de les claus `\texttt{\{}' i fer que els blocs fossin definits per la seva propia indentació, com a Python. Per fer-ho, hem ampliat el \texttt{lexer} per tal de que emeti tokens relacionats amb la indentació (\texttt{INDENT} i \texttt{DEDENT}), tot utilitzant una pila que manté el nombre d'espais per cada nivell d'indentació definit:

\lstinputlisting[style=antlr, firstline=33, lastline=93]{../src/hal/parser/Hal.g}

\noindent La indentació només pot succeïr quan hi ha un salt de línia. Per tant,
afegim el comportament d'analitzar la indentació al token \texttt{NEWLINE}:

\lstinputlisting[style=antlr, firstline=304, lastline=336]{../src/hal/parser/Hal.g}

\noindent Per la sintaxi ens hem basat en llenguatges com Ruby, Smalltalk, Python
i Haskell. Volem que el llenguatge sigui còmode d'escriure, encara que
la gramàtica sembli una mica més complicada:

\lstinputlisting[style=antlr, firstline=114, lastline=247]{../src/hal/parser/Hal.g}

\noindent La part dels tokens és la següent:

\lstinputlisting[style=antlr, firstline=250, lastline=302]{../src/hal/parser/Hal.g}

\clearpage

\section{Exemples}
\subsection{Condicionals}
\lstinputlisting{examples/conditionals.in}
\begin{figure}[h!]
  \includegraphics[width=1\textwidth]{examples/conditionals.pdf}
  \caption{Conditionals. AST resultant. Fes zoom per veure la imatge}
\end{figure}

\clearpage

\subsection{Definició de funcions}
\lstinputlisting{examples/fundefs.in}
\begin{figure}[h!]
  \includegraphics[width=0.9\textwidth]{examples/fundefs.pdf}
  \caption{Definició de funcions. AST resultant}
\end{figure}

\clearpage

\subsection{Crides a funcions}
\lstinputlisting{examples/funcalls.in}
\begin{figure}[h!]
  \includegraphics[width=1\textwidth]{examples/funcalls.pdf}
  \caption{Crides a funcions. AST resultant}
\end{figure}

\clearpage

\subsection{Bucles}
\lstinputlisting{examples/loops.in}
\begin{figure}[h!]
  \includegraphics[width=1\textwidth]{examples/loops.pdf}
  \caption{Bucles. AST resultant}
\end{figure}

\clearpage

\subsection{Llistes}
\lstinputlisting{examples/lists.in}
\begin{figure}[h!]
  \includegraphics[width=1\textwidth]{examples/lists.pdf}
  \caption{Llists. AST resultant}
\end{figure}

\clearpage

\section{Breu descripció dels següents passos}
El següent pas és, principalment, construir un evaluador de l'arbre sintàctic.
En ell haurem de tenir en compte coses com els \emph{scopes} de les variables
i la taula de símbols per a les funcions. Ens podem basar una mica en l'\asl.

Una particularitat és que hem considerat que els identificadors
pertanyin tots a crides de funcions, considerant les \emph{variables}
casos particulars de funcions amb 0 arguments, i que retornen el valor
assignat. Creiem que el fet que tot és comporti com crides a funcions ens permetrà
fer un evaluador més senzill, perquè serà més homogeni.

A \hal{} volem donar-li força importància a les funcions, per
facilitar la generació de \emph{DSL}s. Hi haurà altres tipus de dades
bàsics (numèrics, llistes,...), volem centrar-nos en el tipus {\tt
  Function}.  Volem suportar \emph{currying} i funcions de primer
ordre, per tant, haurem de tenir una classe {\tt Function} força
potent.

També voldriem implementar classes. Per encapsular conjunts de
funcions. Es cridarien amb la notació del {\tt .} (p.e. {\tt instancia.metode
  arg1, arg2}). Si arribéssim a suportar herència segurament seria
només simple, per facilitar l'ordre de resolució de mètodes.

La idea és que a partir d'una base petita del llenguatge, amb un evaluador
senzill, sigui possible construir la resta del \hal{} a sobre.

\end{document}
