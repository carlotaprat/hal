\documentclass{beamer}
\usepackage[utf8]{inputenc}
\usepackage{listings}
\usepackage{xcolor}
\usepackage{graphicx}
\definecolor{dkgreen}{rgb}{0,0.6,0}
\definecolor{gray}{rgb}{0.5,0.5,0.5}
\definecolor{mauve}{rgb}{0.58,0,0.82}
\newcommand{\CodeSymbol}[1]{\textcolor{brown}{#1}}
\lstdefinelanguage{hal}{
    basicstyle=\ttfamily,
    sensitive=true,
    morestring=[b]\",
    morestring=[b]\',
    keywords=[1]{for,in,while,if,else,elif,class,def,with,return,import,from},
    keywords=[2]{Array, print, self, yield, block_given?},
    keywordstyle=[1]\color{blue},
    keywordstyle=[2]\color{red!60!black},
    ={\color{brown!50!black}\bfseries},
    morecomment=[l]{\#},
    commentstyle={\color{gray}\itshape},
    showstringspaces=false,
    stringstyle=\color{green!50!black},
    captionpos=b,
    literate=%
      {\{}{{\CodeSymbol{\{}}}1
      {\}}{{\CodeSymbol{\}}}}1
      {(}{{\CodeSymbol{(}}}1
      {)}{{\CodeSymbol{)}}}1
      {>}{{\CodeSymbol{>}}}1
      {<}{{\CodeSymbol{<}}}1
      {=}{{\CodeSymbol{=}}}1
      {;}{{\CodeSymbol{;}}}1
      {[}{{\CodeSymbol{[}}}1
      {]}{{\CodeSymbol{]}}}1
      {:}{{\CodeSymbol{:}}}1
      {.}{{\CodeSymbol{.}}}1
      {+}{{\CodeSymbol{+}}}1
      {-}{{\CodeSymbol{-}}}1
      {`}{{\CodeSymbol{\`{}}}}1
      {@}{{\CodeSymbol{@}}}1
}

\lstdefinelanguage{output}{
    basicstyle=\ttfamily\footnotesize,
    columns=fullflexible,
    backgroundcolor=\color{gray!180!black},
    xleftmargin=0.5cm,frame=tlbr,framesep=4pt,framerule=0pt
}

\lstdefinestyle{antlr}{
    basicstyle=\scriptsize\ttfamily,
    language=Java,
    moredelim=[s][\color{green!50!black}\ttfamily]{'}{'},
    moredelim=*[s][\bf]{options}{\}},
    emph=[1]{lexer,header,parser},
    emphstyle=[1]{\color{green!70!black}\ttfamily},
    commentstyle={\color{gray}\itshape},
    alsoletter={:,|,;},
    morekeywords={:,|,;},
    keywordstyle=\color{blue},
    stringstyle=\color{mauve},
    breaklines=true,
    postbreak=\raisebox{0ex}[0ex][0ex]{\ensuremath{\color{red}\hookrightarrow\space}}
}

\lstdefinestyle{java}{
  language=Java,
  basicstyle={\scriptsize\ttfamily},
  keywordstyle=\color{blue},
  commentstyle=\color{dkgreen},
  stringstyle=\color{mauve},
  breaklines=true,
  postbreak=\raisebox{0ex}[0ex][0ex]{\ensuremath{\color{red}\hookrightarrow\space}}
}

\usetheme{Madrid}
\title{HAL}
\titlegraphic{
\includegraphics[scale=0.15]{hal.png}
}
\author{Héctor Ramón and Alvaro Espuña}
\date{4th june, 2014}
\begin{document}
\frame{\titlepage}
\begin{frame}[fragile]{Objective}
Create a programming language:
\begin{itemize}
\item
That results useful in the future
\item
Easy to write and read
\item
Very dynamic
\item
Inspired in good ideas of other languages (\texttt{Python}, \texttt{Ruby},...)
\end{itemize}
\begin{figure}[h!]
\begin{lstlisting}[language=hal]
5.times: print "Hello world!"
\end{lstlisting}
\begin{lstlisting}[language=output]
Hello world!
Hello world!
Hello world!
Hello world!
Hello world!
\end{lstlisting}
\label{hello5}
\end{figure}
\end{frame}
\begin{frame}{Main features}
\begin{itemize}
\item
\textbf{Clean} syntax, perfect for creating \textbf{D}omain-\textbf{S}pecific \textbf{L}anguages
\item
\textbf{Object-oriented} with \textbf{inheritance}
\item
\textbf{Dynamic typing} and \textbf{duck typing}
\item
\textbf{Builtin} methods that can be \textbf{overriden} in \texttt{HAL} itself
\item
Module \textbf{imports}
\item
\textbf{First-class} functions
\item
Interactive and easy-to-extend interpreter
\end{itemize}
\end{frame}
\begin{frame}[fragile]{Clean syntax}
\begin{figure}[h!]
\begin{lstlisting}[language=hal]
class Array:
  def sort!:
    return self if size < 2
    p = pop
    filter{x -> x <= p}.sort! ++ [p] ++ filter{x -> x > p}.sort!

a = [1, 3, 2, -1, -2, -3, 20, 40, 1, 2, 200, -5]
print a.sort!
\end{lstlisting}
\begin{lstlisting}[language=output]
[-5, -3, -2, -1, 1, 1, 2, 2, 3, 20, 40, 200]
\end{lstlisting}
\label{quicksort}
\end{figure}
\end{frame}
\begin{frame}[fragile]{\emph{List comprehension}}
\begin{figure}[h!]
\begin{lstlisting}[language=hal]
class Array:
  def map:
    [ yield x for x in self ]

a = range(10).map with x: x * 2
print a
\end{lstlisting}
\begin{lstlisting}[language=output]
[0, 2, 4, 6, 8, 10, 12, 14, 16, 18]
\end{lstlisting}
\label{list_comprehension}
\end{figure}
\end{frame}
\begin{frame}[fragile]{First-class functions}
\begin{figure}[h!]
\begin{lstlisting}[language=hal]
def lambda: &yield

add = lambda with x, y: x + y

print add 5, add 3, add 1, 2
print &add.arity
\end{lstlisting}
\begin{lstlisting}[language=output]
11
2
\end{lstlisting}
\label{first_class_methods}
\end{figure}
\end{frame}
\begin{frame}[fragile]{Lambda blocks}
\begin{figure}[h!]
\begin{lstlisting}[language=hal]
class Integer:
  def times:
    i = 0
    while i < self:
      yield if &yield.arity == 0 else yield i
      i = i + 1

5.times: print "Something"
3.times with t: print "%s" % t
\end{lstlisting}
\begin{lstlisting}[language=output]
Something
Something
Something
Something
Something
0
1
2
\end{lstlisting}
\label{blocks_times}
\end{figure}
\end{frame}
\begin{frame}[fragile]{\emph{Backticks}}
\begin{figure}[h!]
\begin{lstlisting}[language=hal]
File.open "test.txt" with f:
  f.write "Writing a file has never been so easy!"

print `cat test.txt`
\end{lstlisting}
\begin{lstlisting}[language=output]
Writing a file has never been so easy!
\end{lstlisting}
\label{backticks}
\end{figure}
\end{frame}
\begin{frame}[fragile]{Spaces are significative}
\begin{figure}[h!]
\begin{lstlisting}[language=hal]
def f x: x * 20
a = 10

print \
  a - 1,
  a-1,
  f -1
\end{lstlisting}
\begin{lstlisting}[language=output]
9
9
-20
\end{lstlisting}
\label{spaces_matter}
\end{figure}
\end{frame}
\begin{frame}{Four levels of scopes}
\begin{description}
\item[Local]
Without \emph{accessor}
\item[Instance]
Using \texttt{@}
\item[Static]
Using \texttt{@@}
\item[Module]
Instance variables in the current module
\end{description}
When a name is referenced without any \emph{accessor}, \texttt{HAL} looks in the scopes in that order.
\end{frame}
\begin{frame}[fragile]{Builtin classes}
\begin{itemize}
\item
Boolean
\item
Class
\item
Enumerable
\begin{itemize}
\item
Array
\item
Dictionary
\item
String
\end{itemize}
\item
File
\item
Kernel
\begin{itemize}
\item
Module
\end{itemize}
\item
None
\item
Number
\begin{itemize}
\item
Float
\item
Integer
\item
Long
\item
Rational
\end{itemize}
\item
Object
\item
Package
\item
Process
\end{itemize}
\end{frame}
\begin{frame}[fragile,breaklines=true]{Basic data types}
\begin{lstlisting}[style=java]

public abstract class HalNumber<T extends Number>
    extends HalObject<T> {
  public abstract HalNumber add(HalNumber n);

  public abstract boolean canCoerce(HalObject n);
  public abstract HalNumber coerce(HalObject n);\end{lstlisting}
\end{frame}
\begin{frame}[fragile]{Basic data types (I)}
\begin{lstlisting}[style=java]

private static final Reference __add__ =
  new Reference(new Builtin("add", new Params.Param("x")) {
    @Override
    public HalObject mcall(HalObject instance,
      HalMethod lambda, Arguments args) {
       HalNumber i = ((HalNumber) instance);
       HalObject x = args.get("x");

            if (!i.canCoerce(x))
                return x.methodcall("__radd__", i);

            return i.add(i.coerce(x));
        }
    });\end{lstlisting}
\end{frame}
\begin{frame}[fragile]{Basic data types (II)}
\begin{lstlisting}[style=java]

public static final HalClass klass =
   new HalClass("Number", HalObject.klass,
  // ...
  __add__,
  // ...
);\end{lstlisting}
\begin{lstlisting}[style=java]

class HalInteger extends HalNumber<Integer> {
 // ...
 @Override
 public HalNumber add(HalNumber n) {
   if (addOverflows(toInteger(), n.toInteger()))
     return new HalLong(toInteger())
            .add(new HalLong(n.toInteger()));
   return new HalInteger(toInteger() + n.toInteger());
 }
 // ...
}\end{lstlisting}
\end{frame}
\begin{frame}{Basic data types (III)}
Other \texttt{Number} methods:
\begin{itemize}
\item[·]
\texttt{bool}
\item[·]
\texttt{pos}
\item[·]
\texttt{neg}
\item[·]
\texttt{sub}
\item[·]
\texttt{mul}
\item[·]
\texttt{pow}
\item[·]
\texttt{div}
\item[·]
\texttt{mod}
\item[·]
\texttt{ddiv}
\item[·]
\texttt{eq}
\item[·]
\texttt{lt}
\end{itemize}
\end{frame}
\begin{frame}[fragile]{\texttt{HALTeX}}
\begin{figure}[h!]
\begin{lstlisting}[language=hal]
import haltex

section 'Use example'
enumerate:
  item; p '*First item*'
  item; p '|second|'
  item; p '**third**'
\end{lstlisting}
\begin{lstlisting}[language=output]
\section{Use example}
\begin{enumerate}
\item
\emph{First item}
\item
\texttt{second}
\item
\textbf{third}
\end{enumerate}
\end{lstlisting}
\label{haltex}
\end{figure}
\end{frame}
\end{document}
