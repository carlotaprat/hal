\documentclass[a4paper,11pt]{article}
\usepackage[utf8]{inputenc}
\usepackage{graphicx}
\usepackage[justification=centering,labelfont=bf]{caption}
\usepackage[hidelinks]{hyperref}
\usepackage[usenames,dvipsnames]{xcolor}
\usepackage{listings}
\lstdefinelanguage{hal}{
    basicstyle=\ttfamily,
    sensitive=false,
    morestring=[b]",
    keywords={=,for,in,while,if,else,elif,class,def},
    keywordstyle=\color{blue},
    morecomment=[l]{\#},
    showstringspaces=false,
    stringstyle=\color{green!50!black},
    captionpos=b
  }
\begin{document}
\begin{titlepage}
\begin{center}
\textsc{\Large Compilers}
\\[1.5cm]
\rule{\linewidth}{0.5mm}
\\[0.4cm]
{\huge
\bfseries
Hal
\\[0.4cm]
}
\rule{\linewidth}{0.5mm}
\\[2.5cm]
\begin{minipage}{0.4\textwidth}
\begin{flushleft}
\large
Héctor Ramón Jiménez
\end{flushleft}
\end{minipage}
\begin{minipage}{0.4\textwidth}
\begin{flushright}
\large
Alvaro Espuña Buxo
\end{flushright}
\end{minipage}
\vfill
{\large
\today
}
\\
{\large
\texttt{Facultat d'Informàtica de Barcelona}
}
\end{center}
\end{titlepage}
\section{Introduction}
HAL is a scripting language based in other scripting languages like Python and Ruby.
    Our objective is to provide a clean, easy-to-use and powerful scripting language with the
    most interesting features and ideas that can be found in other languages!
\begin{figure}[h!]
\begin{lstlisting}[language=hal]
print "Hello world!"
\end{lstlisting}
\begin{verbatim}
=> Hello world!
\end{verbatim}
\caption{Hal says: ``Hello world!"}
\end{figure}
\end{document}
